% vim: spelllang=en_gb

\documentclass[12pt,a4paper,abstract=true,draft]{scrartcl}
\usepackage{ifdraft}

% --------------------
% Set Language Options
% --------------------

\usepackage[nswissgerman,french,main=english]{babel}
\usepackage[autostyle,english=american,german=swiss]{csquotes}
\MakeOuterQuote{"}

\usepackage[shortcuts]{extdash}

% --------------
% Font & Symbols
% --------------

\usepackage{amssymb,mathtools}
\usepackage[warnings-off={mathtools-colon,mathtools-overbracket}]{unicode-math}
\usepackage[oldstyle,proportional]{libertinus-otf}

% ---------------
% Set Page Layout
% ---------------

% Get length of 65 characters
%\setlxvchars

\usepackage[driver=auto]{geometry}
% A5: 148mm × 210mm
% A4: 210mm × 297mm
\geometry{
  width=140mm,
  height=217mm,
  marginparsep=3mm,
  marginparwidth=30mm,
}
\ifdraft{\geometry{
  inner=10mm,
  marginparwidth=50mm
}}{}


% ---------------------
% Load Various Packages
% ---------------------

% Various Math Environments
\usepackage{amsthm,thmtools}

% Bibliography
\usepackage[style=numeric-comp,url=false,isbn=false,maxnames=6]{biblatex}
\addbibresource{bibliography.bib}

% For general figures
\usepackage[final]{graphicx}
\graphicspath{{/img}}
\usepackage{subcaption}
\usepackage{tikz}
\usetikzlibrary{babel,cd,shapes,3d}
\tikzcdset{arrow style=math font}
\tikzset{cross/.style={
    cross out, draw, solid, thin, 
    minimum size=2*(#1-\pgflinewidth), 
    inner sep=0pt, outer sep=0pt
  },
  cross/.default={3}
}

% For lists
\usepackage[shortlabels]{enumitem}
\setlist{nosep}

% For better Tables
%\usepackage{tabularray}

% For more fine grained typesetting in final mode.
% Else set the tolerance for overfull warnings higher.
\ifdraft{\hfuzz=1.5pt}{\usepackage{microtype}}

% Links and stuff
\usepackage[final]{hyperref}
\usepackage[noabbrev,capitalize]{cleveref}

% Suppress Latex build info in PDF (for uploads to arxiv)
\hypersetup{
  pdfcreator = {},
  pdfproducer = {}
}
\pdfvariable suppressoptionalinfo \numexpr 1+2+4+8+16+32+64+128+256+512 \relax


% For Todonotes
\usepackage[obeyDraft]{luatodonotes}

% --------------------------------------------
% Define Theorem Environments & Math Operators
% --------------------------------------------

\declaretheorem[numberwithin=section]{theorem}
\declaretheorem[sibling=theorem]{lemma, proposition, corollary}
\declaretheorem[sibling=theorem,style=definition]{definition, example}
\declaretheorem[sibling=theorem,style=remark]{remark}
\declaretheorem[name=Theorem,
refname={Theorem,Theorems},
Refname={Theorem,Theorems}]{maintheorem}
\renewcommand{\themaintheorem}{\Alph{maintheorem}}


\DeclareMathOperator{\id}{id}
\DeclareMathOperator{\im}{im}
\DeclareMathOperator{\interior}{int}
\DeclareMathOperator{\Aut}{Aut}
\DeclareMathOperator{\Diff}{Diff}
\DeclareMathOperator{\GL}{GL}
\DeclareMathOperator{\HF}{HF}
\DeclareMathOperator{\HM}{HM}
\DeclareMathOperator{\Hom}{Hom}
\DeclareMathOperator{\Ext}{Ext}
\DeclareMathOperator{\Tor}{Tor}
\DeclareMathOperator{\Flux}{Flux}
\DeclareMathOperator{\Crit}{Crit}

% ----------------------------
% Various Marcos and Shortcuts
% ----------------------------

\usepackage{physics2} % various shortcuts
\usephysicsmodule{ab,ab.legacy}
\newcommand\mqty[1]{\begin{pmatrix}#1\end{pmatrix}}

\begin{document}
\title{Branch Cut Tangles Part I}
\author{Joel Schmitz}
\maketitle

\section{Tangles Replace Mutations}

We want to give an exposition of the papers ideas in the $4$-dimensional case.
To do so, we will introduce a slightly different viewpoint on the concepts of base diagram and nodal slide, which is well-suited for generalization later in \todo{THE FUUUUTARE}.

Let $X$ be a 4-dimensional symplectic manifold and $π \colon X → B$ be an almost toric fibration.
As described in \cite{evans2021atfs}, there is a set $N ⊂ B$ of isolated nodes, such that for every $n ∈ N$, the fibre $π^{-1}(n)$ contains a exactly one\footnote{This is a (technically unnecessary) assumption, which mainly serves to produce easier to read diagrams later on.} focus-focus critical point of $π$.
The manifold $B ∖ N$ carries an integral affine structure given by the integrable system, which gives a $ℤ²$-lattice bundle $Λ ⊂ T(B ∖ N)$.

\subsection{The monodromy pair}
Take $n ∈ N$ and let $U_n$ be a contractible open neighbourhood of $n$ containing no other nodes.
By \cite[Proposition 1]{Zun97}, there is a unique (up to sign) faithful $S¹$-action on $U_n$, generated by $f ∘ π$ for some $f ∈ 𝒞^∞(U_n)$.
Write $λ = df$.
Since $f$ generates a faithful $S¹$-action, $λ ∈ Λ^* ⊂ T^* U_n$ is a primitive covector field.

Choose a generator $a$ of $H_1(U_n ∖ \{n\}) ≅ ℤ$. (This is equivalent to choosing a coorientation on the point $n$.\todo{…})
For any point $x ∈ U_n ∖ \{n\}$, this induces the monodromy map $M_x \colon Λ_x → Λ_x$, given by sweeping an element $u ∈ Λ_x$ along a representative of $a$ with endpoints in $x$.
As described in \cite[Lemma 1]{Zun97}, choosing an appropriate basis $M_x$ can be written as
\[ M_x = \mqty{1 & 1\\0 & 1}\; ,\]
and has a unique eigenspace $V_x ⊂ T_x U_n$ of eigenvalue $1$, such that $V_x = \ker λ_x$.
Now there is a unique primitive vector field $v ∈ Γ(U_n,Λ)$ such that $v_x ∈ V_x$ and $M_x = \id + v_x ⟨ λ_x, · ⟩$.
We call the pair $(v,λ)$ a \textbf{monodromy pair of $n$}, and $v$ and $λ$ \textbf{monodromy (co-)vector field} respectively.

The role of $λ$ is somewhat auxiliary in the case of $2$-dimensional bases we are interested in: If there is a (local) orientation on $B$, then we can write $λ_x(·) = v_x ∧ ·$.

Note that once the coorientation is fixed (for example by a possibly existing global orientation on $B$), the monodromy pair $(v,λ)$ is unique up to sign.
It is also independent of the choice of $U_n$ in the sense that if $V_n$ is another open contractible neighbourhood of $n$ with no other nodes, then there exists an open contractible neighbourhood $W_n ⊂ U_n ∩ V_n$ of $n$ such that the monodromy pairs of $U_n$ and $V_n$ coincide on $W_n$ up to sign.
Conversely if $(v,λ)$ is defined on $V_n ⊂ U_n$, then $(v,λ)$ extends uniquely to $U_n$.
Thus the data $(v,λ)$ is purely local.

\subsection{Nodal Slides}
A nodal slide allows one to locally modify the fibration $π$ such that a node $n$ slides along its monodromy vector field:

\begin{lemma}[Nodal Slide]
  \label{thm:nodal_slide}
  Let $n ∈ N$ be a node, take $U_n$ as above and $(v,λ)$ a monodromy pair for $n$.
  Take $f ∈ 𝒞^∞_c(U_n)$ supported on a compact set $C ⊂ U_n$, and $φ_t$ the time $t$-flow of $fv$.
  Then there is a smooth family of almost toric fibrations $π_t \colon X → B$ such that
  \begin{enumerate}
    \item $φ_t(n)$ is the only node of $π_t|_{U_n}$.
    \item $π_0 = π$
    \item $π|_{B ∖ C} = π_t|_{B ∖ C}$
  \end{enumerate}
\end{lemma}

See \cite[Theorem 6.5]{symington2002FourDF} or \cite[Theorem 8.10]{evans2021atfs} for a proof.

\subsection{Nodal charts \& Tangles}

We adopt a different notion of base diagram than used in e.g. \cite{symington2002FourDF,evans2021atfs}.
While it serves a similar purpose, it is sufficiently different to merit its own name:

\begin{definition}
  Let $U ⊂ B$ be open, connected.
  A \textbf{nodal chart} of $U$ is a homeomorphism $φ \colon U → ℝ²$ that is piecewise integral affine on $U ∖ N$.
\end{definition}

The piecewise integral affine condition ensures that for a point $x ∈ \im φ$ in the integral affine locus, $π ∘ φ$ gives action coordinates in a neighbourhood of $(φ ∘ π)^{-1}(x)$.
More generally, for any point $x$ in $φ(B ∖ N)$, we may regard it as the origin of $ℝ²$.
Then there exists an invertible piecewise integral \emph{linear} map $S_x\colon ℝ² → ℝ²$ such that $S_x ∘ φ$ is integral affine near $x$, i.e. $S_x ∘ φ ∘ π$ gives action coordinates in a neighbourhood of $(φ ∘ π)^{-1}(x)$. (On the integral affine parts we may choose $S_x = \id$.)
We call $S_x$ a \textbf{resolution map at $x$}.
It is unique up to composing with an integral linear map.

The non-affine locus of $φ$ in $\im φ$ is the union of closed line segments\footnote{Here we include closed rays and lines in the definition of closed line segment.}, which we call \textbf{branch cuts}.
We can think of it as a "graph"\footnote{In the same sense that a tropical curve is a graph. I.e. there may be edges going outside of $\im φ$ with no vertex at the end. However it does not satisfy the balancing condition of tropical curves.} $G$, where the edges are branch cuts.
We used quotation marks as we also allow rays and lines to be "edges" of the "graph".
If the only vertices of $G$ are leaves (i.e. nodes or vertices on $φ(∂B ∩ U)$), we call $φ$ \textbf{tangle-free}.

\begin{remark}
  \label{rem:bc_resolution}
  Suppose $x$ is a point in the interior of a branch cut $𝓁$.
  Then its resolution map $S_x$ is given by two integral affine maps that agree on $𝓁$.
W.l.o.g.\ we may take one of them to be the identity.
Then the other one must fix $𝓁$, and thus is an integral shear map, so $S_x$ must be equal to the \textbf{half-shear}
\begin{equation}
  \label{eq:half_shear}
  S_v^k(y) = \begin{cases}
    y - k v\det(v,y) & \text{if } \det(v,y) ≥ 0 \\
    y & \text{if } \det(v,y) ≤ 0
  \end{cases} \; ,
\end{equation}
where $v$ is a primitive vector along $𝓁$ (corresponding to an orientation of $𝓁$), and $k$ a positive integer.
Requiring $k$ to be positive induces an orientation on $𝓁$, given by $v$.
\end{remark}

\begin{example}[From tangle-free to tangled]
  \begin{figure}
    \centering
    \begin{tikzpicture}[scale=1.2]
      \begin{scope}[shift={(0,0)}]
        \fill[opacity=0.1] (-1,1) rectangle (2.5,-1);
        \draw[red,->] (-1,-1) -- (-0.5,-0.5) node[anchor=west] {$a$};
        \draw[green!50!black,->] (-1,1) -- (-0.5,0.5) node[anchor=west] {$b$};
        \draw[blue,->] (2.5,-1) -- (2,-0.5) node[anchor=east] {$c$};
        \fill (0,0) circle[radius=.7pt] node[anchor=north west] {$x$};
        \draw[thick] (-1,1) rectangle (2.5,-1);
        \node at (.75,-1) [anchor=north] {a)};
      \end{scope}
      \begin{scope}[shift={(4,0)}]
        \fill[opacity=0.1] (-1,1) rectangle (2.5,-1);
        \draw[red,->] (-1,-1) -- (0.5,0.5) node[anchor=south] {$a$};
        \draw[green!50!black,->] (-1,1) -- (-0.5,0.5) node[anchor=west] {$b$};
        \draw[blue,->] (2.5,-1) -- (2,-0.5) node[anchor=east] {$c$};
        \fill (0,0) circle[radius=.7pt] node[anchor=north west] {$x$};
        \draw[thick] (-1,1) rectangle (2.5,-1);
        \node at (.75,-1) [anchor=north] {b)};
      \end{scope}
      \begin{scope}[shift={(8,0)}]
        \fill[opacity=0.1] (-1,1) rectangle (2.5,-1);
        \draw[red,->] (-1,-1) -- (0.5,0.5) node[anchor=south] {$a$};
        \draw[green!50!black,->] (-1,1) -- (-.03,.03) (.1,.03) -- (.9,.3)  node[anchor=west] {$b$};
        \draw[blue,->] (2.5,-1) -- (2,-0.5) node[anchor=east] {$c$};
        \fill (0,0) circle[radius=.7pt] node[anchor=north west] {$x$};
        \draw[thick] (-1,1) rectangle (2.5,-1);
        \node at (.75,-1) [anchor=north] {c)};
      \end{scope}
      
      \begin{scope}[shift={(0,-3)}]
        \fill[opacity=0.1] (-1,1) rectangle (2.5,-1);
        \draw[red,->] (-1,-1) -- (0,0)  (.3,.06) -- (2,0.4) node[anchor=south] {$a$};
        \draw[red,densely dashed] (0,0) -- (0.5,0.5) arc (135:-45:.03) -- (0.15+.04,.15-0.04);
        \draw[green!50!black,->] (-1,1) -- (-.03,.03) (.1,.03) -- (.9,.3)  node[anchor=south] {$b$};
        \draw[blue,->] (2.5,-1) -- (2,-0.5) node[anchor=east] {$c$};
        \fill (0,0) circle[radius=.7pt] node[anchor=north west] {$x$};
        \draw[thick] (-1,1) rectangle (2.5,-1);
        \node at (.75,-1) [anchor=north] {d)};
      \end{scope}
      \begin{scope}[shift={(4,-3)}]
        \fill[opacity=0.1] (-1,1) rectangle (2.5,-1);
        \draw[red,->] (-1,-1) -- (0.5,0.5) arc (135:-45:.03) -- (0.15+.04,.15-0.04) (.3,.06) -- (2,0.4);
        \draw[green!50!black,->] (-1,1) -- (-.03,.03) (.1,.03) -- (.9,.3);
        \draw[blue,->] (2.5,-1) -- (1.5-0.22,0.22) (1.25,0.25) ++(0.11,0.04) -- ++(1.1,0.4);
        \draw[thick] (-1,1) rectangle (2.5,-1);
        \node at (.75,-1) [anchor=north] {e)};
      \end{scope}
      \begin{scope}[shift={(8,-3)}]
        \node at (.75,0) {…};
      \end{scope}
    \end{tikzpicture}
    \caption{Introducing tangles into a tangle-free nodal chart.}
    \label{fig:building_tangles}
  \end{figure}
  This example shall motivate the proceeding and following definitions.

  Take a look at \cref{fig:building_tangles} a).
  There are three branch cuts, drawn as coloured lines.
  They have endpoints in three nodes $a,b,c$, marked by arrow tips.
  This diagram provides action coordinates everywhere except on the nodes and the branch cuts.

  Since $φ$ is a homeomorphism (meaning the diagram "closes up" as in \cite[Section 7.2]{evans2021atfs}, the direction of the branch cuts ending in the nodes corresponds to the nodes' monodromy vector field, with the arrow tip indicating its direction.


  Let's modify $π$ and $φ$ by nodal slides to introduce a few "tangles": First slide $a$ such that its branch cut is "in front of" $b$, ass seen in \cref{fig:building_tangles} b).
  In doing so we modified the integral affine structure along $a$'s branch cut. 
  The resolution map at the marked point $x$ is now no longer trivial since $x$ sits on a branch cut.
  Concretely, let $v_1 = \mqty{1\\1}$, then $S_x = S_{v_1}$, with $S_{v_1}$ as in \eqref{eq:half_shear}.

  Moving on, we can slide $b$ along its eigendirection, past the point $x$.
  As our nodal chart $φ$ as a "break" in the integral affine structure described by the resolution map $S_x$, the straight line segment in $B$ along which we slide $b$ appears to bend in $\im φ$ at the point $x$.
  Concretely the incoming vector $v_2' = \mqty{1\\-1}$ exits $x$ as $v_2 = \mqty{3\\1}$.(One can be determined from the other by the condition $S_x(-v_2') = -S_x(v_2)$, which expresses that the nodal slide should be straight with respect to the integral affine structure.)
  We get \cref{fig:building_tangles} c).
  Since we modified the integral affine structure around $φ^{-1}(x)$, the resolution map at $x$ once again changed: It now is the composition $S_{v_2} ∘ S_{v_1}$.

  Sliding $a$ back through $x$, it is once again deflected, giving \cref{fig:building_tangles} d).
As in the paragraph above we determine the outgoing vector $v_3$ pointing from $x$ to the node $a$ to be $v_3 = \mqty{5\\1}$, giving the new resolution map $S_x = S_{v_3} ∘ S_{v_2} ∘ S_{v_1}$.
  The dashed part of the red line hooking back into $x$ is only there to illustrate which order of nodal slides we took to get to diagram d), not an actual branch cut.
  However we will usually also draw it solidly.
  
  Proceeding in the same manner we can continue to tangle up our fibration $π$ and nodal chart $φ$.
\end{example}



\begin{definition}
  A \textbf{branch cut tangle} is a nodal chart $φ \colon B ⊃ U → ℝ²$ where all branch cuts and non-leave vertices of $G$ are labelled as follows:
  \begin{enumerate}
    \item A branch cut $𝓁$ is labelled $(k,v)$, such that for $x ∈ 𝓁$ in the interior of $𝓁$, the resolution map is given by $S^k_v$ as given in \eqref{eq:half_shear}.
    \item A non-leave vertex $x$ is labelled by a list $((k_1,v_1), …, (k_n,v_n))$ such that $S_x = S^{k_n}_{v_n} ∘ … ∘ S^{k_1}_{v_1}$.
      We call such a point with its list a \textbf{local tangle}.\todo{Or better branch cut stack?}
  \end{enumerate}
\end{definition}

\begin{remark}
  Obvious question: Is every nodal chart a branch cut tangle? I.e. can every piecewise integral \emph{linear} map be decomposed as a series of half-shears?.
\end{remark}

\begin{remark}
  Tangles vs.\ mutations...
\end{remark}

\section{Displacement energy of fibres in toric manifolds.}

We will use a bit of tropical geometry in this section.
If you are unfamiliar with the topic, you may wish to read \cite[Sections 1-2]{BruSha14}, which gives a nice overview of the needed language.
For $x,y ∈ ℝ ∪ \{+∞\}$ we write $x ⊕ y ≔ \min\{x,y\}$ for the tropical addition and $x y = x+y$ for tropical multiplication.
For $x ∈ ℝ²$ and $λ ∈ ℤ²$ we write $x^{λ} = x_1^{λ_1}  x_2^{λ_2} = ⟨λ,x⟩$.
To avoid confusion we will try to never mix classical and tropical notation.

Let $Δ ⊂ ℝ²$ be a Delzant polytope, and $μ \colon (X,ω) → Δ$ the corresponding toric symplectic manifold identified by Delzant's theorem \cite{Del88}.

If $Δ$ has $n ∈ ℕ ∪ \{∞\}$ edges, we can write
\[Δ = \{x ∈ ℝ² \mid \min_{0 ≤ i < n}\{𝓁_i(x)\} ≥ 0 \} \; ,\]
where $𝓁_i = ⟨λ_i, ·⟩ + c_i, λ_i ∈ ℤ², c_i ∈ ℝ$ are integral affine functions $ℝ² → ℝ$ defining the edges of $Δ$, satisfying the Delzant condition, i.e. if $x ∈ ∂Δ$ and $𝓁_i(x) = 𝓁_j(x) = 0$, then $\det(λ_i,λ_j) = ±1$.
Reinterpreting $\min_{0 ≤ i < n}\{𝓁_i(x)\}$ as a tropical polynomial $⨁_{i=0}^n c_i  x^{λ_i}$ motivates the following definitions:

\begin{definition}
  Let $Q$ be a tropical polynomial in $2$ variables.
  For $a ∈ ℝ$, its \textbf{$a$-level polygon} is $Δ_Q^a = Q^{-1}(ℝ_{≥a})$.
  We call its $0$-level polygon $Δ_Q^0$ simply its polygon and write $Δ_Q ≔ Δ_Q^0$.
\end{definition}

To a tropical polynomial $Q$ we associate its tropical curve $C_Q$, which divides $Δ_Q$ into faces, each corresponding the locus where $Q$ is equal to one of its monomial terms.
An edge of $C_Q$ is contained in the locus of coincidence of two monomial terms $c_1  x^{λ_1}, c_2  x^{λ_2}$.
We identify an edge by the exponents of these two terms $(λ_1,λ_2)$.

\begin{definition}
  An edge $(λ_1,λ_2)$ of $C_Q$ is called \textbf{Delzant} if $\abs{\det(λ_1,λ_2)} = 1$.
\end{definition}
In particular, a Delzant edge has weight $1$.

\begin{definition}
  Take $a ∈ ℝ$.
  $Q$ is called $a$-Delzant if
  \begin{enumerate}
    \item All monomials of $Q$ have primitive exponents.
    \item All edges of $C_Q$ intersecting $∂Δ_Q^a$ are Delzant.
    \item $∂Δ_Q^a$ contains no vertices of $C_Q$.
    \item No face of $C_Q$ is entirely contained in $Δ_Q^a$.
  \end{enumerate}
  If $a=0$ we simply say $Q$ is Delzant.
\end{definition}

Properties 2 \& 3 ensure that $Δ_Q^a$ is a Delzant polygon.
Property 1 \& 4 ensures that given a Delzant polygon $Δ$ and $a ∈ ℝ$, there is a unique $Q$ such that $Q$ is $a$-Delzant and $Δ=Δ_Q^a$, up to monomial terms which don't influence $C_Q ∩ Δ_Q^a$.

\begin{remark}
  Property 1 might seem superfluous, as property 2 implies all exponents contributing to an edge intersecting $∂Δ$ are primitive.
  The only case where property 1 is relevant is $Q(x) = ax^{ λ} ⊕ bx^{-λ}$, corresponding to the symplectic toric manifold $S² × T^*S¹$, where we require $λ$ to be primitive to ensure uniqueness.
\end{remark}

\begin{proposition}
  \label{thm:smoothness}
  Suppose $Q$ is Delzant and $C_Q$ only has vertices of degree $3$ inside $Δ_Q$ (generic).
  Set $M = \sup_{y ∈ ℝ²}\{Q(y)\}$.
  Then all vertices of $C_Q$ contained in $Δ_Q ∖ Δ_Q^M$ have multiplicity $1$, and all edges of $C_Q$ in $Δ_Q ∖ Δ_Q^M$ are Delzant.

  Furthermore, $Δ_Q^M$ is either empty, a point consisting of a multiplicity $3$ vertex of $C_Q$ or a line segment corresponding to a weight $2$ edge of $C_Q$ whose endpoints (if any) are vertices of $C_Q$ of multiplicity $2$.
\end{proposition}


\begin{proof}
  Sort the vertices $\{v_1,…,v_m\}$ of $C_Q$ contained in $Δ_Q ∖ Δ_Q^M$ by their "height" $Q(v_i)$ in ascending order.
  We will proceed by induction on this list to show that $v_i$ has multiplicity $1$ and all edges contained in $Δ_Q ∖ Δ_Q^{Q(v_i+1)}$ are Delzant.

  Let $v$ be the lowest vertex in that list.
  Let $c_0  x^{λ_0},c_1  x^{λ_1},c_2  x^{λ_2}$ be the three monomial terms of $Q$ incident at $v$.
  After a translation we may assume $v=0 ∈ ℝ²$ and $c_0 = c_1 = c_2$.

  By the Delzant property 4 of $Q$, none of the three faces bordering $v$ can be entirely contained in $Δ_Q^{Q(v)}$.
  Consequently one of the faces must be contained in $Δ_Q ∖ Δ_Q^{Q(v)}$, as well as two of the incident edges at $v$ of $C_Q$.
  By Delzant property 2 (or induction), these edges are Delzant, meaning that (after reshuffling indices) $\det(λ_0,λ_1) = \det(λ_1,λ_2) = 1$.

  Set $k = \det(λ_0,λ_2)$.
  After an $GL(2,ℤ)$ transform, we may write
  \[ λ_0 = \mqty{0 & -1}, λ_1 = \mqty{1 & 0}, λ_2 = \mqty{k & 1} \;. \]
  If $k=1$ then $v$ has multiplicity $1$ and all incident edges at $v$ are Delzant.

  If $k = 2$, the edges $(λ_0,λ_1)$ and $(λ_1,λ_2)$ are parallel, contradicting the existence of $v$.
  If $k > 2$ the $λ_1$-face has a minimum at $v$ and thus would be contained in $Δ_Q^{Q(v)}$, contradicting Delzant property 3.

  If $k ≤ 0$, then $v$ is a maximum of $c_0  x^{λ_0} ⊕ c_1  x^{λ_1} ⊕ c_2  x^{λ_2}$, and thus also of $Q$, so $v ∈ Δ_Q^M$.

  Proceeding by induction we prove the first part of the lemma.

  For the structure of $Δ_Q^M$, it suffices to inspect the case $k ≤ 0$ above a bit closer:
  If $k<-1$, the edge $(λ_0,λ_2)$ is not Delzant and is contained in $Δ_Q ∖ Δ_Q^M$, contradicting the first part of the lemma.
  If $k=-1$, then $v$ has multiplicity $3$ and is the only maximum of $Q$, so $Δ_Q^M = \{v\}$.
  If $k=0$, then $v$ has multiplicity $2$ and the edge $(λ_0,λ_2)$ is $Δ_Q^M$.
  The case where $Δ_Q^M$ is a line without ends occurs when $Q(x) = ax^{λ} ⊕ bx^{-λ}$.
\end{proof}

\begin{corollary}
  Let $Q$ be $a$-Delzant.
  Then $Q$ is $b$-Delzant for all $b>a$ such that $∂Δ_Q^b$ contains no vertices of $C_Q$.
\end{corollary}

\begin{lemma}
  \label{thm:tropical_line}
  If $v$ is a vertex of $C_Q$ with multiplicity $1$ and all its incident edges are Delzant, then $Q$ is locally given by $c(x_1 ⊕ x_1 x_2 ⊕ x_2)$ up to a $GL(2,ℤ)$ transform.
\end{lemma}

\begin{proof}
  Follows from proof of \cref{thm:smoothness}.
\end{proof}

\begin{theorem}
  \label{thm:displacing}
  Let $Q$ be Delzant and $μ\colon X → Δ_Q$ the corresponding toric symplectic manifold.
  Set $M = \sup_{y ∈ Δ_Q}\{Q(y)\}$ and take $x ∈ Δ_Q ∖ C_Q$ with $Q(x) < \frac{1}{2} M$.
  Then the displacement energy of $μ^{-1}(x)$ is $Q(x)$, and there exists a almost toric fibration $π\colon X → B$ and a probe $P ⊂ B$ that displaces $μ^{-1}(x)$ with energy $Q(x)$.
\end{theorem}

Probes where introduces by McDuff in \cite{mcduff2011displacing} in the context of toric manifolds.
Here we take a probe in an almost toric fibration to be a straight line segment $P ⊂ B ∖ N$ s.t.\ $P$ intersects the toric boundary $∂B$ integrally transversely, $P$ has no self intersections.
That way we can choose action-coordinates on a contractible neighbourhood $P$ where one can perform the symplectic reduction.

\begin{proof}[Proof of \cref{thm:displacing}]
  $e(μ^{-1}(x)) ≥ Q(x)$ by Delzant construction see \todo{…}.
  
  We now construct the almost toric fibration $π$.
  Modify $μ$ by nodal trades in every toric corner to get $π$, and take a nodal chart $φ\colon B → Δ_Q$ such that $μ = φ ∘ π$ outside of small neighbourhoods of the toric fixed points of $μ$.
  Note that the branch cuts at each node point along the edges of $C_Q$ intersecting $∂Δ_Q$.

  Perform nodal slides until the nodes reach a vertex $v$ of $C_Q$.
  If $v ∈ Δ_Q^M$, stop.
  Otherwise, by \cref{thm:smoothness}, $v$ has multiplicity $1$, all incident edges of $C_Q$ are Delzant and there are two nodes incident at $v$.
  By \cref{thm:tropical_line}, after a $GL(2,ℤ)$ transform, the faces incident at $v$ are given by 
  \[λ_0 = \mqty{1 & 0} \;, λ_1 = \mqty{1 & 1} \;, λ_2 = \mqty{0 & 1}\;.\]
  The incident nodes $n_1,n_2$ then have monodromy vector fields of
  \[v_1 = \mqty{1\\0} \;, v_2 = \mqty{0\\1} \;, \]
  given in the local basis given by $φ$.
  Slide $n_1$ through $v$, and then $n_2$.
  After sliding $n_2$ through $v$, its monodromy vector field will be given by
  \[v_2' = \mqty{1\\1} \; ,\]
  so $n_2$ now slides along the edge $(λ_0,λ_2)$ of $C_Q$.
  Note that $n_1$ now slides "horizontally" with respect to $Q$, i.e.\ sliding $n_1$ does not change the value of $Q(φ(n_1))$.

  Proceed in this way to "fill-up" $C_Q ∩ (Δ_Q ∖ Δ_Q^M)$ with branch cuts.

  Thus we obtain an almost toric fibration and nodal chart
  \[\begin{tikzcd}
      X \ar[r,"π"] & B \ar[r,"φ"] & Δ_Q
    \end{tikzcd}
    \; .\]

  Take any $x ∈ Δ_Q ∖ C_Q$.
  Since $x$ is on a face of $C_Q$, the tropical polynomial $Q$ is locally given by a monomial $c_0 x^{λ_0}$.
  The idea to displace $x$ is the following: Take any direction $p_0$ at $x$ such that $⟨λ_0,p_0⟩ = 1$.
  Now with respect to the integral affine structure of $B$ take a straight line $P$ passing through $φ^{-1}(x)$ in the direction $p$.
  This will be a suitable probe.
  Indeed every time $P$ crosses from one face of $C_Q$ in $Δ_Q$ into another, $φ$ has a branch cut there ensuring that $⟨λ,p⟩ = 1$ also holds on the new face.

  Indeed, take a point on an edge $(λ_1,λ_2)$ of $C_Q ∖ Δ_Q^M$, and assume $φ(P)$ is crossing $(λ_1,λ_2)$ with direction $p_1$ in the $λ_1$-face and $p_2$ in the $λ_2$-face.
  Since $(λ_1,λ_2)$ is Delzant by \cref{thm:smoothness}, after a $GL(2,ℤ)$ transform
  \begin{align*}
    λ_1 &= \mqty{1 & 0} , & λ_2 &= \mqty{-1 & 1} \\
    p_1 &= \mqty{1 \\ k} , & p_2 &= \mqty{1-k \\ k} \; ,
  \end{align*}
  where we have used that the branch cut of $φ$ over $(λ_1,λ_2)$ has resolution map $S_v$, where $v = \mqty{1 & 0}$.

  Descending along $P$ from $x$ takes us to $∂Δ_Q$ along a path of affine length $Q(x)$.
  Ascending along $P$ from $x$ takes us arbitrarily close to $Δ_Q^M$ along a path of affine length $M-Q(x)-ε$, for some small $ε>0$.

  Thus we have found a probe $P$ of length $M-ε$ for arbitrarily small $ε$ displacing $μ^{-1}(x) = (φ∘π)^{-1}(x)$, since $Q(x) < \frac{1}{2}M$.
\end{proof}

\end{document}
