% vim: spelllang=en_gb

\documentclass[12pt,a4paper,abstract=true,draft]{scrartcl}
\usepackage{ifdraft}

% --------------------
% Set Language Options
% --------------------

\usepackage[nswissgerman,french,main=english]{babel}
\usepackage[autostyle,english=american,german=swiss]{csquotes}
\MakeOuterQuote{"}

\usepackage[shortcuts]{extdash}

% --------------
% Font & Symbols
% --------------

\usepackage{amssymb,mathtools}
\usepackage[warnings-off={mathtools-colon,mathtools-overbracket}]{unicode-math}
\usepackage[oldstyle,proportional]{libertinus-otf}

% ---------------
% Set Page Layout
% ---------------

% Get length of 65 characters
%\setlxvchars

\usepackage[driver=auto]{geometry}
% A5: 148mm × 210mm
% A4: 210mm × 297mm
\geometry{
  width=140mm,
  height=217mm,
  marginparsep=3mm,
  marginparwidth=30mm,
}
\ifdraft{\geometry{
  inner=10mm,
  marginparwidth=50mm
}}{}


% ---------------------
% Load Various Packages
% ---------------------

% Various Math Environments
\usepackage{amsthm,thmtools}

% Bibliography
\usepackage[style=numeric-comp,url=false,isbn=false,maxnames=6]{biblatex}
\addbibresource{bibliography.bib}

% For general figures
\usepackage[final]{graphicx}
\graphicspath{{/img}}
\usepackage{subcaption}
\usepackage{tikz}
\usetikzlibrary{babel,cd,shapes,3d}
\tikzcdset{arrow style=math font}
\tikzset{cross/.style={
    cross out, draw, solid, thin, 
    minimum size=2*(#1-\pgflinewidth), 
    inner sep=0pt, outer sep=0pt
  },
  cross/.default={3}
}

% For lists
\usepackage[shortlabels]{enumitem}
\setlist{nosep}

% For better Tables
%\usepackage{tabularray}

% For more fine grained typesetting in final mode.
% Else set the tolerance for overfull warnings higher.
\ifdraft{\hfuzz=1.5pt}{\usepackage{microtype}}

% Links and stuff
\usepackage[final]{hyperref}
\usepackage[noabbrev,capitalize]{cleveref}

% Suppress Latex build info in PDF (for uploads to arxiv)
\hypersetup{
  pdfcreator = {},
  pdfproducer = {}
}
\pdfvariable suppressoptionalinfo \numexpr 1+2+4+8+16+32+64+128+256+512 \relax


% For Todonotes
\usepackage[obeyDraft]{luatodonotes}

% --------------------------------------------
% Define Theorem Environments & Math Operators
% --------------------------------------------

\declaretheorem[numberwithin=section]{theorem}
\declaretheorem[sibling=theorem]{lemma, proposition, corollary}
\declaretheorem[sibling=theorem,style=definition]{definition, example}
\declaretheorem[sibling=theorem,style=remark]{remark}
\declaretheorem[name=Theorem,
refname={Theorem,Theorems},
Refname={Theorem,Theorems}]{maintheorem}
\renewcommand{\themaintheorem}{\Alph{maintheorem}}


\DeclareMathOperator{\id}{id}
\DeclareMathOperator{\im}{im}
\DeclareMathOperator{\interior}{int}
\DeclareMathOperator{\Aut}{Aut}
\DeclareMathOperator{\Diff}{Diff}
\DeclareMathOperator{\GL}{GL}
\DeclareMathOperator{\HF}{HF}
\DeclareMathOperator{\HM}{HM}
\DeclareMathOperator{\Hom}{Hom}
\DeclareMathOperator{\Ext}{Ext}
\DeclareMathOperator{\Tor}{Tor}
\DeclareMathOperator{\Flux}{Flux}
\DeclareMathOperator{\Crit}{Crit}

% ----------------------------
% Various Marcos and Shortcuts
% ----------------------------

\usepackage{physics2} % various shortcuts
\usephysicsmodule{ab,ab.legacy}
\newcommand\mqty[1]{\begin{pmatrix}#1\end{pmatrix}}

\begin{document}
\title{Branch Cut Tangles Part I}
\author{Joel Schmitz}
\maketitle

\section{Tangles Replace Mutations}

We want to give an exposition of the papers ideas in the $4$-dimensional case.
To do so, we will introduce a slightly different viewpoint on the concepts of base diagram and nodal slide, which is well-suited for generalization later in \todo{THE FUUUUTARE}.

Let $X$ be a 4-dimensional symplectic manifold and $π \colon X → B$ be an almost toric fibration.
As described in \cite{evans2021atfs}, there is a set $N ⊂ B$ of isolated nodes, such that for every $n ∈ N$, the fibre $π^{-1}(n)$ contains a exactly one\footnote{This is a (technically unnecessary) assumption, which mainly serves to produce easier to read diagrams later on.} focus-focus critical point of $π$.
The manifold $B ∖ N$ carries an integral affine structure given by the integrable system, which gives a $ℤ²$-lattice bundle $Λ ⊂ T(B ∖ N)$.

\paragraph{The monodromy pair}
Take $n ∈ N$ and let $U_n$ be a contractible open neighbourhood of $n$ containing no other nodes.
By \cite[Proposition 1]{Zun97}, there is a unique (up to sign) faithful $S¹$-action on $U_n$, generated by $f ∘ π$ for some $f ∈ 𝒞^∞(U_n)$.
Write $λ = df$.
Since $f$ generates a faithful $S¹$-action, $λ ∈ Λ^* ⊂ T^* U_n$ is a primitive covector field.

Choose a generator $a$ of $H_1(U_n ∖ \{n\}) ≅ ℤ$. (This is equivalent to choosing a coorientation on the point $n$.\todo{…})
For any point $x ∈ U_n ∖ \{n\}$, this induces the monodromy map $M_x \colon Λ_x → Λ_x$, given by sweeping an element $u ∈ Λ_x$ along a representative of $a$ with endpoints in $x$.
As described in \cite[Lemma 1]{Zun97}, choosing an appropriate basis $M_x$ can be written as
\[ M_x = \mqty{1 & 1\\0 & 1}\; ,\]
and has a unique eigenspace $V_x ⊂ T_x U_n$ of eigenvalue $1$, such that $V_x = \ker λ_x$.
Now there is a unique primitive vector field $v ∈ Γ(U_n,Λ)$ such that $v_x ∈ V_x$ and $M_x = \id + v_x ⟨ λ_x, · ⟩$.
We call the pair $(v,λ)$ a \textbf{monodromy pair of $n$}, and $v$ and $λ$ \textbf{monodromy (co-)vector field} respectively.

The role of $λ$ is somewhat auxiliary in the case of $2$-dimensional bases we are interested in: If there is a (local) orientation on $B$, then we can write $λ_x(·) = v_x ∧ ·$.

Note that once the coorientation is fixed (for example by a possibly existing global orientation on $B$), the monodromy pair $(v,λ)$ is unique up to sign.
It is also independent of the choice of $U_n$ in the sense that if $V_n$ is another open contractible neighbourhood of $n$ with no other nodes, then there exists an open contractible neighbourhood $W_n ⊂ U_n ∩ V_n$ of $n$ such that the monodromy pairs of $U_n$ and $V_n$ coincide on $W_n$ up to sign.
Conversely if $(v,λ)$ is defined on $V_n ⊂ U_n$, then $(v,λ)$ extends uniquely to $U_n$.
Thus the data $(v,λ)$ is purely local.

\paragraph{Nodal Slides}
A nodal slide allows one to locally modify the fibration $π$ such that a node $n$ slides along its monodromy vector field:

\begin{lemma}[Nodal Slide]
  \label{thm:nodal_slide}
  Let $n ∈ N$ be a node, take $U_n$ as above and $(v,λ)$ a monodromy pair for $n$.
  Take $f ∈ 𝒞^∞_c(U_n)$ supported on a compact set $C ⊂ U_n$, and $φ_t$ the time $t$-flow of $fv$.
  Then there is a smooth family of almost toric fibrations $π_t \colon X → B$ such that
  \begin{enumerate}
    \item $φ_t(n)$ is the only node of $π_t|_{U_n}$.
    \item $π_0 = π$
    \item $π|_{B ∖ C} = π_t|_{B ∖ C}$
  \end{enumerate}
\end{lemma}

See \cite[Theorem 6.5]{symington2002FourDF} or \cite[Theorem 8.10]{evans2021atfs} for a proof.

\paragraph{Base diagrams \& Tangles}

We adopt a different notion of base diagram than used in e.g. \cite{symington2002FourDF,evans2021atfs}.
While it serves a similar purpose, it is sufficiently different to merit its own name:

\begin{definition}
  Let $U ⊂ B$ be open, connected.
  A \textbf{nodal chart} of $U$ is a homeomorphism $φ \colon U → ℝ²$ that is piecewise integral affine on $U ∖ N$.
\end{definition}

The piecewise integral affine condition ensures that for a point $x ∈ \im φ$ in the integral affine locus, $π ∘ φ$ gives action coordinates in a neighbourhood of $(φ ∘ π)^{-1}(x)$.
More generally, for any point $x$ in $φ(B ∖ N)$, there exists an invertible piecewise integral affine map $S_x\colon ℝ² → ℝ²$ such that $S_x ∘ φ$ is integral affine near $x$, i.e. $S_x ∘ φ ∘ π$ gives action coordinates in a neighbourhood of $(φ ∘ π)^{-1}(x)$. (On the integral affine parts we may choose $S_x = \id$.)
We call $S_x$ a \textbf{resolution map at $x$}.
It is unique up to composing with an integral affine map.

The non-affine locus of $φ$ in $\im φ$ is the union of closed line segments\footnote{Here we include closed rays and lines in the definition of closed line segment.}, which we call \textbf{branch cuts}.
We can think of it as a "graph" $G$\todo{Is this a tropical curve?}, where the edges are branch cuts.
We used quotation marks as we also allow rays and lines to be "edges" of the "graph".
If the only vertices of $G$ are leaves (i.e. nodes or on vertices on $φ(∂B ∩ U)$), we call $φ$ \textbf{tangle-free}.

\begin{remark}
  \label{rem:bc_resolution}
  Suppose $x$ is a point in the interior of a branch cut $𝓁$. W.l.o.g.\ we assume $x = 0 ∈ ℝ²$.
Then its resolution map $S_x$ is given by two integral affine maps that agree on $𝓁$.
W.l.o.g.\ we may take one of them to be the identity.
Then the other one must fix $𝓁$, and thus is an integral shear map, so $S_x$ must be equal to the "half-shear"
\begin{equation}
  \label{eq:half_shear}
  S_v^k(y) = \begin{cases}
    y - k v\det(v,y) & \text{if } \det(v,y) ≥ 0 \\
    y & \text{if } \det(v,y) ≤ 0
  \end{cases} \; ,
\end{equation}
where $v$ is a primitive vector along $𝓁$ (corresponding to an orientation of $𝓁$), and $k$ a positive integer.
\end{remark}

\begin{example}[From tangle-free to tangled]
  \begin{figure}
    \centering
    \begin{tikzpicture}[scale=1.2]
      \begin{scope}[shift={(0,0)}]
        \fill[opacity=0.1] (-1,1) rectangle (2.5,-1);
        \draw[red,->] (-1,-1) -- (-0.5,-0.5) node[anchor=west] {$a$};
        \draw[green!50!black,->] (-1,1) -- (-0.5,0.5) node[anchor=west] {$b$};
        \draw[blue,->] (2.5,-1) -- (2,-0.5) node[anchor=east] {$c$};
        \fill (0,0) circle[radius=.7pt] node[anchor=north west] {$x$};
        \draw[thick] (-1,1) rectangle (2.5,-1);
        \node at (.75,-1) [anchor=north] {a)};
      \end{scope}
      \begin{scope}[shift={(4,0)}]
        \fill[opacity=0.1] (-1,1) rectangle (2.5,-1);
        \draw[red,->] (-1,-1) -- (0.5,0.5) node[anchor=south] {$a$};
        \draw[green!50!black,->] (-1,1) -- (-0.5,0.5) node[anchor=west] {$b$};
        \draw[blue,->] (2.5,-1) -- (2,-0.5) node[anchor=east] {$c$};
        \fill (0,0) circle[radius=.7pt] node[anchor=north west] {$x$};
        \draw[thick] (-1,1) rectangle (2.5,-1);
        \node at (.75,-1) [anchor=north] {b)};
      \end{scope}
      \begin{scope}[shift={(8,0)}]
        \fill[opacity=0.1] (-1,1) rectangle (2.5,-1);
        \draw[red,->] (-1,-1) -- (0.5,0.5) node[anchor=south] {$a$};
        \draw[green!50!black,->] (-1,1) -- (-.03,.03) (.1,.03) -- (.9,.3)  node[anchor=west] {$b$};
        \draw[blue,->] (2.5,-1) -- (2,-0.5) node[anchor=east] {$c$};
        \fill (0,0) circle[radius=.7pt] node[anchor=north west] {$x$};
        \draw[thick] (-1,1) rectangle (2.5,-1);
        \node at (.75,-1) [anchor=north] {c)};
      \end{scope}
      
      \begin{scope}[shift={(0,-3)}]
        \fill[opacity=0.1] (-1,1) rectangle (2.5,-1);
        \draw[red,->] (-1,-1) -- (0,0)  (.3,.06) -- (2,0.4) node[anchor=south] {$a$};
        \draw[red,densely dashed] (0,0) -- (0.5,0.5) arc (135:-45:.03) -- (0.15+.04,.15-0.04);
        \draw[green!50!black,->] (-1,1) -- (-.03,.03) (.1,.03) -- (.9,.3)  node[anchor=south] {$b$};
        \draw[blue,->] (2.5,-1) -- (2,-0.5) node[anchor=east] {$c$};
        \fill (0,0) circle[radius=.7pt] node[anchor=north west] {$x$};
        \draw[thick] (-1,1) rectangle (2.5,-1);
        \node at (.75,-1) [anchor=north] {d)};
      \end{scope}
      \begin{scope}[shift={(4,-3)}]
        \fill[opacity=0.1] (-1,1) rectangle (2.5,-1);
        \draw[red,->] (-1,-1) -- (0.5,0.5) arc (135:-45:.03) -- (0.15+.04,.15-0.04) (.3,.06) -- (2,0.4);
        \draw[green!50!black,->] (-1,1) -- (-.03,.03) (.1,.03) -- (.9,.3);
        \draw[blue,->] (2.5,-1) -- (1.5-0.22,0.22) (1.25,0.25) ++(0.11,0.04) -- ++(1.1,0.4);
        \draw[thick] (-1,1) rectangle (2.5,-1);
        \node at (.75,-1) [anchor=north] {e)};
      \end{scope}
      \begin{scope}[shift={(8,-3)}]
        \node at (.75,0) {…};
      \end{scope}
    \end{tikzpicture}
    \caption{Introducing tangles into a tangle-free nodal chart.}
    \label{fig:building_tangles}
  \end{figure}
  This example shall motivate the proceeding and following definitions.

  Take a look at \cref{fig:building_tangles} a).
  There are three branch cuts, drawn as coloured lines.
  They have endpoints in three nodes $a,b,c$, marked by arrow tips.
  This diagram provides action coordinates everywhere except on the nodes and the branch cuts.

  Since $φ$ is a homeomorphism (meaning the diagram "closes up" as in \cite[Section 7.2]{evans2021atfs}, the direction of the branch cuts ending in the nodes corresponds to the nodes' monodromy vector field $v$, with the arrow tip indicating its direction.


  Let's modify $π$ and $φ$ by nodal slides to introduce a few "tangles": First slide $a$ such that its branch cut is "in front of" $b$, ass seen in \cref{fig:building_tangles} b).
  In doing so we modified the integral affine structure along $a$'s branch cut. 
  The resolution map at the marked point $x$ is now no longer trivial since $x$ sits on a branch cut.
  Concretely, let $v_1 = \mqty{1\\1}$, then $S_x = S_{v_1}$, with $S_{v_1}$ as in \eqref{eq:half_shear}.

  Moving on for now, we can slide $b$ along its eigendirection, past the point $x$.
  As our nodal chart $φ$ as a "break" in the integral affine structure described by the resolution map $S_x$, the straight line segment in $B$ along which we slide $b$ appears to bend in $\im φ$ at the point $x$.
  Concretely the incoming vector $v_2' = \mqty{1\\-1}$ exits $x$ as $v_2 = \mqty{3\\1}$.(One can be determined from the other by the condition $S_x(-v_2') = -S_x(v_2)$, which expresses that the nodal slide should be straight with respect to the integral affine structure.)
  We get \cref{fig:building_tangles} c).
  Since we modified the integral affine structure around $φ^{-1}(x)$, the resolution map at $x$ once again changed: It now is the composition $S_{v_2} ∘ S_{v_1}$.

  Sliding $a$ back through $x$, it is once again deflected, giving \cref{fig:building_tangles} d).
As in the paragraph above we determine the outgoing vector $v_3$ pointing from $x$ to the node $a$ to be $v_3 = \mqty{5\\1}$, giving the new resolution map $S_x = S_{v_3} ∘ S_{v_2} ∘ S_{v_1}$.
  The dashed part of the red line hooking back into $x$ is only there to illustrate which order of nodal slides we took to get to diagram d), not an actual branch cut.
  However we will usually also draw it solidly.
  
  Proceeding in the same manner we can continue to tangle up our $π$ and $φ$...
\end{example}



\begin{definition}
  A \textbf{branch cut tangle} is a nodal chart $φ \colon B ⊃ U → ℝ²$ where all branch cuts and non-leave vertices of $G$ are labelled as follows:
  \begin{enumerate}
    \item A branch cut is labelled $(k,v)$, such that for $x ∈ 𝓁$ in the interior of $𝓁$, $S_x = S^k_v$ as given in \eqref{eq:half_shear}.
    \item A non-leave vertex $x$ is labelled by a list $((k_1,v_1), …, (k_n,v_n))$ such that $S_x = S^{k_n}_{v_n} ∘ … ∘ S^{k_1}_{v_1}$.
      We call such a point with its list a \textbf{local tangle}.\todo{Or better branch cut stack?)}
  \end{enumerate}
\end{definition}

\begin{remark}
  Obvious question: Is every nodal chart a branch cut tangle? I.e. can every piecewise integral \emph{linear} map be decomposed as a series of half-shears?.
\end{remark}

\section{Displacement energy of fibres in toric manifolds.}

We will use a bit of tropical geometry in this section.
If you are unfamiliar with the topic, you may wish to read \cite[Sections 1-2]{BruSha14}, which gives a nice overview of the needed language.
For $x,y ∈ ℝ ∪ \{+∞\}$ we write $x ⊕ y ≔ \min{x,y}$ for the tropical addition and $x ⊗ y = x+y$ for tropical multiplication.
For $x ∈ (ℝ ∪ \{+∞\})²$ and $λ ∈ ℤ²$ we write $x^{⊗λ} = x_1^{⊗λ_1} ⊗ x_2^{⊗λ_2} = ⟨λ,x⟩$.

Let $μ \colon (X,ω) → Δ ⊂ ℝ²$ be a toric manifold with moment map $μ$.
According to the Atiyah-Guillemin-Sternberg convexity theorem \cite{Ati82,GuiSte82}, we can write
\[Δ = \{x ∈ ℝ² \mid \min_{1 ≤ i ≤ n}\{𝓁_i(x)\} ≥ 0 \} \; ,\]
where $n ≥ 0$ and $𝓁_i = ⟨λ_i, ·⟩ + c_i, λ_i ∈ ℤ², c_i ∈ ℝ$ are integral affine functions $ℝ² → ℝ$, satisfying the Delzant condition, i.e. if $x ∈ ∂Δ$ and $𝓁_i(x) = 𝓁_j(x) = 0$, then $\det(λ_i,λ_j) = ±1$.
In other words, there exists a tropical polynomial in two variables
\[Q(x) = ⨁_{1 ≤ i ≤ n} c_i ⊗ x^{⊗λ_i}\]
such that $Δ = \{Q(x) ≥ 0\}$ and all edges of the tropical curve $C$ given by $Q$ which intersect $∂Δ = \{Q(x) = 0\}$ have weight $1$.

\begin{theorem}
  Let $x ∈ Δ ∖ C$.
  Then $e(μ^{-1}(x)) = Q(x)$, and there exists a almost toric fibration $π\colon X → B$ and a symmetric probe $P ⊂ B$ that displaces $μ^{-1}(x)$ with energy $Q(x)$.
\end{theorem}

Probes where introduces by McDuff in \cite{mcduff2011displacing} in the context of toric manifolds.
Here we take a probe in an almost toric fibration to be a straight line segment $P ⊂ B ∖ N$ s.t.\ $P$ intersects the toric boundary $∂B$ integrally transversely, $P$ has no self intersections.
That way we can choose action-coordinates on a contractible neighbourhood $P$ where one can perform the symplectic reduction.

\end{document}
