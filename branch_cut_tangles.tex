% vim: spelllang=en_gb

\documentclass[12pt,a4paper,abstract=true,draft]{scrartcl}
\usepackage{ifdraft}

% --------------------
% Set Language Options
% --------------------

\usepackage[nswissgerman,french,main=english]{babel}
\usepackage[autostyle,english=american,german=swiss]{csquotes}
\MakeOuterQuote{"}

\usepackage[shortcuts]{extdash}

% --------------
% Font & Symbols
% --------------

\usepackage{amssymb,mathtools}
\usepackage[warnings-off={mathtools-colon,mathtools-overbracket}]{unicode-math}
\usepackage[oldstyle,proportional]{libertinus-otf}

% ---------------
% Set Page Layout
% ---------------

% Get length of 65 characters
%\setlxvchars

\usepackage[driver=auto]{geometry}
% A5: 148mm × 210mm
% A4: 210mm × 297mm
\geometry{
  width=140mm,
  height=217mm,
  marginparsep=3mm,
  marginparwidth=30mm,
}
\ifdraft{\geometry{
  inner=10mm,
  marginparwidth=50mm
}}{}


% ---------------------
% Load Various Packages
% ---------------------

% Various Math Environments
\usepackage{amsthm,thmtools}

% Bibliography
\usepackage[style=numeric-comp,url=false,isbn=false,maxnames=6]{biblatex}
\addbibresource{bibliography.bib}

% For general figures
\usepackage[final]{graphicx}
\graphicspath{{/img}}
\usepackage{subcaption}
\usepackage{tikz}
\usetikzlibrary{babel,cd,shapes,3d}
\tikzcdset{arrow style=math font}
\tikzset{cross/.style={
    cross out, draw, solid, thin, 
    minimum size=2*(#1-\pgflinewidth), 
    inner sep=0pt, outer sep=0pt
  },
  cross/.default={3}
}

% For lists
\usepackage[shortlabels]{enumitem}
\setlist{nosep}

% For better Tables
%\usepackage{tabularray}

% For more fine grained typesetting in final mode.
% Else set the tolerance for overfull warnings higher.
\ifdraft{\hfuzz=1.5pt}{\usepackage{microtype}}

% Links and stuff
\usepackage[final]{hyperref}
\usepackage[noabbrev,capitalize]{cleveref}

% Suppress Latex build info in PDF (for uploads to arxiv)
\hypersetup{
  pdfcreator = {},
  pdfproducer = {}
}
\pdfvariable suppressoptionalinfo \numexpr 1+2+4+8+16+32+64+128+256+512 \relax


% For Todonotes
\usepackage[obeyDraft]{luatodonotes}

% --------------------------------------------
% Define Theorem Environments & Math Operators
% --------------------------------------------

\declaretheorem[numberwithin=section]{theorem}
\declaretheorem[sibling=theorem]{lemma, proposition, corollary}
\declaretheorem[sibling=theorem,style=definition]{definition, example}
\declaretheorem[sibling=theorem,style=remark]{remark}
\declaretheorem[name=Theorem,
refname={Theorem,Theorems},
Refname={Theorem,Theorems}]{maintheorem}
\renewcommand{\themaintheorem}{\Alph{maintheorem}}


\DeclareMathOperator{\id}{id}
\DeclareMathOperator{\im}{im}
\DeclareMathOperator{\interior}{int}
\DeclareMathOperator{\Aut}{Aut}
\DeclareMathOperator{\Diff}{Diff}
\DeclareMathOperator{\GL}{GL}
\DeclareMathOperator{\HF}{HF}
\DeclareMathOperator{\HM}{HM}
\DeclareMathOperator{\Hom}{Hom}
\DeclareMathOperator{\Ext}{Ext}
\DeclareMathOperator{\Tor}{Tor}
\DeclareMathOperator{\Flux}{Flux}
\DeclareMathOperator{\Crit}{Crit}

% ----------------------------
% Various Marcos and Shortcuts
% ----------------------------

\usepackage{physics2} % various shortcuts
\usephysicsmodule{ab,ab.legacy}
\newcommand\mqty[1]{\begin{pmatrix}#1\end{pmatrix}}

\begin{document}
\title{Branch Cut Tangles}
\author{Joel Schmitz}
\maketitle

\section{Tangles Replace Mutations}

We want to give an exposition of the papers ideas in the $4$-dimensional case.
To do so, we will introduce a slightly different viewpoint on the concepts of base diagram and nodal slide, which is well-suited for generalization later in \cref{sec:general_atfs}.

Let $X$ be a 4-dimensional symplectic manifold and $π \colon X → B$ be an almost toric fibration.
As described in \cite{evans2021atfs}, there is a set $N ⊂ B$ of isolated nodes, such that for every $n ∈ N$, the fibre $π^{-1}(n)$ contains a exactly one\footnote{This is a (technically unnecessary) assumption, which mainly serves to produce easier to read diagrams later on.} focus-focus critical point of $π$.
The manifold $B ∖ N$ carries an integral affine structure given by the integrable system, which gives a $ℤ²$-lattice bundle $Λ ⊂ T(B ∖ N)$.

\paragraph{The monodromy pair}
Take $n ∈ N$ and let $U_n$ be a contractible open neighbourhood of $n$ containing no other nodes.
By \cite[Proposition 1]{Zun97}, there is a unique (up to sign) faithful $S¹$-action on $U_n$, generated by $f ∘ π$ for some $f ∈ 𝒞^∞(U_n)$.
Write $λ = df$.
Since $f$ generates a faithful $S¹$-action, $λ ∈ Λ^* ⊂ T^* U_n$ is a primitive covector field.

Choose a generator $a$ of $H_1(U_n ∖ \{n\}) ≅ ℤ$. (This is equivalent to choosing a coorientation on the point $n$.\todo{…})
For any point $x ∈ U_n ∖ \{n\}$, this induces the monodromy map $M_x \colon Λ_x → Λ_x$, given by sweeping an element $u ∈ Λ_x$ along a representative of $a$ with endpoints in $x$.
As described in \cite[Lemma 1]{Zun97}, choosing an appropriate basis $M_x$ can be written as
\[ M_x = \mqty{1 & 1\\0 & 1}\; ,\]
and has a unique eigenspace $V_x ⊂ T_x U_n$ of eigenvalue $1$, such that $V_x = \ker λ_x$.
Now there is a unique primitive vector field $v ∈ Γ(U_n,Λ)$ such that $v_x ∈ V_x$ and $M_x = \id + v_x ⟨ λ_x, · ⟩$.
We call the pair $(v,λ)$ a \textbf{monodromy pair of $n$}, and $v$ and $λ$ \textbf{monodromy (co-)vector field} respectively.

The role of $λ$ is somewhat auxiliary in the case of $2$-dimensional bases we are interested in: If there is a (local) orientation on $B$, then we can write $λ_x(·) = v_x ∧ ·$.

Note that once the coorientation is fixed (for example by a possibly existing global orientation on $B$), the monodromy pair $(v,λ)$ is unique up to sign.
It is also independent of the choice of $U_n$ in the sense that if $V_n$ is another open contractible neighbourhood of $n$ with no other nodes, then there exists an open contractible neighbourhood $W_n ⊂ U_n ∩ V_n$ of $n$ such that the monodromy pairs of $U_n$ and $V_n$ coincide on $W_n$ up to sign.
Conversely if $(v,λ)$ is defined on $V_n ⊂ U_n$, then $(v,λ)$ extends uniquely to $U_n$.
Thus the data $(v,λ)$ is purely local.

\paragraph{Nodal Slides}
A nodal slide allows one to locally modify the fibration $π$ such that a node $n$ slides along its monodromy vector field:

\begin{lemma}[Nodal Slide]
  \label{thm:nodal_slide}
  Let $n ∈ N$ be a node, take $U_n$ as above and $(v,λ)$ a monodromy pair for $n$.
  Take $f ∈ 𝒞^∞_c(U_n)$ supported on a compact set $C ⊂ U_n$, and $φ_t$ the time $t$-flow of $fv$.
  Then there is a smooth family of almost toric fibrations $π_t \colon X → B$ such that
  \begin{enumerate}
    \item $φ_t(n)$ is the only node of $π_t|_{U_n}$.
    \item $π_0 = π$
    \item $π|_{B ∖ C} = π_t|_{B ∖ C}$
  \end{enumerate}
\end{lemma}

See \cite[Theorem 6.5]{symington2002FourDF} or \cite[Theorem 8.10]{evans2021atfs} for a proof.

\paragraph{Base diagrams \& Tangles}
We adopt a slightly different notion of base diagram than used in e.g. \cite{symington2002FourDF,evans2021atfs}:

\begin{definition}
  Let $U ⊂ B$ be open. A \textbf{base diagram} of $U$ is a homeomorphism $φ \colon U → ℝ²$ that is piecewise linear on $U ∖ N$.
\end{definition}


\section{(Simple) Almost Toric Fibrations}
\label{sec:general_atfs}

As mentioned in \cite{symington2002FourDF,evans2021atfs}, the theory of almost toric fibrations can be generalized to higher dimensions.
In this section we want to give a brief outline of a simple generalization which is enough for our purposes.

\begin{definition}
  An \textbf{integrable system} is a fibration $π\colon (X,ω) → B$ such that $\{π^*f,π^*g \} = 0$ for all open sets $U ⊂ B$ and $f,g ∈ 𝒞^∞(U)$.
  \todo{$X_{π^*f}$ linearly independent?}
\end{definition}

\begin{definition}
  For any closed $ξ ∈ Ω¹(B)$ we define its symplectic vector field $X_ξ$ by $ω(X_ξ, ·) = π^* ξ$, and denote the time-$t$ flow along $X_ξ$ by $φ_ξ^t$ if it exists.
\end{definition}

\begin{definition}
  The \textbf{affine monodromy sheaf} $𝒫_π$ of $π \colon X → B$ is defined by
  \[𝒫_π(U) ≔ \{ ξ ∈ Ω¹(U) \mid dξ=0, φ^1_ξ = \id\} \; .\]
\end{definition}

If there is no confusion which integrable system is discussed, we write $𝒫 = 𝒫_π$.
We may think of elements of $𝒫(U)$ as system-preserving $S¹$-actions.
This was first introduced in \cite[§3.3]{Zun03}.
(We use the letter $𝒫$ instead of $ℛ$ in analogy with the period lattice, see e.g.\ \cite[Section 1.4]{evans2021atfs}.
Another suitable name for $𝒫$ would be "period sheaf".)

\begin{definition}
  A \textbf{(simple) almost toric fibration} is an integrable system $π\colon (X,ω) → B$ with connected compact fibres, non-degenerate\todo{definieren} elliptic and topologically stable\todo{definieren} non-degenerate codimension 2\todo{definieren} focus-focus singularities.
  The points in the set $N ⊂ B$ of fibres containing focus-focus singularities are called nodes.
\end{definition}

Let $π \colon (X,ω) → B$ be an almost toric fibration, and $N ⊂ B$ the set of nodes.

\begin{remark}
  By \cite[Proposition 5.4 b)]{Zun96} $N$ is a $(n-2)$-dimensional submanifold of $B$.
\end{remark}

Let $U ⊂ B$ be an open contractible set containing only regular fibres or elliptic singular fibres.
Then $𝒫(U) = ℤ^n$.
In the case of only regular fibres this follows from the classical Arnold-Liouville Theorem, in the case with elliptic singularities it follows from e.g.\ \cite{DuMo91}.
We may thus view $𝒫$ as an integral affine structure on $B ∖ N$, and its stalks give the \textbf{dual integral lattice} $Λ_x^* = 𝒫_x ⊂ T_x^*(B ∖ N)$ and the \textbf{integral lattice} $Λ_x = 𝒫_x^* ⊂ T_x(B ∖ N)$.
A basis $(ξ_1,…,ξ_n)$ of $𝒫(U)$ can be written as the differentials of $(f_1,…,f_n) \colon U → ℝ^n$, locally defining an integral affine chart on $U$.
Then we may choose $(α_1,…,α_n) \colon X → T^n$ such that $(f_1∘π, … , f_n∘π, α_1, …, α_n)$ are action-angle coordinates as in the Arnold-Liouville Theorem.

\iffalse
More generally if $(f_1,…,f_n)$ are local coordinates on $B$ (not necessarily with $df_i ∈ 𝒫(U)$), then for any Lagrangian section $σ\colon O → X$ there are $α_1,…,α_n \colon X → ℝ^n$ such that $(f_1 ∘π,…,f_n ∘π, α_1,…,α_n)$ are Liouville coordinates on a neighbourhood of $σ(B)$.
These satisfy
\begin{align*}
  \{f_iπ,α_j\} &= δ_{ij} & \{f_iπ,f_jπ\} &= 0 & \{α_i,α_j\} = 0 \; .
\end{align*}
In particular $π_* X_{α_i} = ∂_{f_i} π$, as $df_i π_* X_{α_j} = d(f_iπ) X_{α_j} = \{f_iπ,α_j\} = δ_{ij}$.
\fi

\subsection{Monodromy}

If $U ⊂ O$ is open and contractible with $N ∩ U$ non-empty and connected, then $𝒫(U) = ℤ^{n-1}$, with a basis of $𝒫(U)$ corresponding to a $T^{n-1}$-action free away from the singular points in $π^{-1}(U)$, see \cite[Theorem 5.2 a)]{Zun96}.
Also by \cite[Theorem 5.2 b)]{Zun96} there is a unique (up to orientation) $S¹$-action $±λ$ in $𝒫(U)$ which fixes the singular points in $π^{-1}(U)$ and is free everywhere else.

The intersection $V = ⋂_{ξ ∈ 𝒫(U)} \ker ξ ⊂ TU$ defines a line-bundle over $U$.
As $U$ is contractible this is a trivial line-bundle.
Take a non-zero section $\tilde{v}$ of $V$.
On $U ∖ N$ we may normalize $\tilde{v}$ to get $v$ so that $v(x) ∈ Λ_x$ is primitive.
Then $v$ is unique up to multiplication by $-1$.

The following is a coordinate-free higher dimensional version of the usual monodromy calculation for focus-focus singularities, see e.g.\ \cite[Lemma 1]{Zun97}.
\begin{lemma}[Monodromy]
  \label{thm:monodromy}
  Let $x ∈ U ∖ N$.
  The Monodromy of $H_1(π^{-1}(x))$ induced by a generator of $π_1(U ∖ N) ≅ ℤ$ is given by
  $\id + v(x) ⟨λ(x), · ⟩$
  for an appropriate choice of sign for $v$ and $λ$.
\end{lemma}
We call $λ$ and $v$ satisfying \cref{thm:monodromy} the monodromy (co)vector field for $N ∩ U$ respectively. (Thus the pair $(v,λ)$ is unique up to multiplication by $-1$.)

\begin{proof}
  By discussion preceding \cite[Definition 5.3]{Zun96}, there exists a free system preserving Hamiltonian $T^{n-2}$-action on $U$.
  With symplectic reduction along this action, we can reduce to the $4$-dimensional case, where the result follows by \cite[Lemma 1]{Zun97}.
\end{proof}

\subsection{Nodal Slides}

Let $π_t \colon X → B$ be a smooth family of smooth maps.
Set $\dot{π}_t = ∂_t π_t$.
\begin{lemma}
  \label{thm:integrable_system_pertubation}
  $π_t$ is an integrable system for all $t$ if and only if $π_0$ is an integrable system and $\{df \dot{π}_t,gπ_t\} + \{f π_t, dg \dot{π}_t\} = 0$ for all $f,g ∈ 𝒞^∞(B)$.
\end{lemma}
\begin{proof}
  Integrate.
\end{proof}

Let $π\colon X → B$ be an almost toric fibration.
Let $U ∈ B$ be a contractible neighbourhood with $U ∩ N$ connected, and let $(f_1,f_2,…,f_n)$ be a chart on $U$ such that $(df_2,…,df_n)$ is a basis of $𝒫(U)$ and $df_2 = λ$ is the monodromy covector field of $U ∩ N$ on $U$.
Let $v$ be the corresponding monodromy vector field on $U ∖ N$.

The following lemma characterises deformations of $π$ that preserve $𝒫(U)$:
\begin{lemma}
  Let $π_t \colon X → B$ be a smooth family of fibrations with $π_0=π$, with $t ∈ I$ where $I$ is an interval.
  Then $π_t$ is an integrable system with $ξ\dot{π}_t = 0 ∈ 𝒞^∞(X)$ for all $ξ ∈ 𝒫(U)$ and $t ∈ I$ if and only if $\dot{π}_t = h_t ·(v∘π)$ where $h_t ∈ 𝒞^∞(X)$ is $ξ$-invariant for all $ξ ∈ 𝒫(U)$.
\end{lemma}
\begin{proof}
  $ξ \dot{π}_t = 0$ means that $\dot{π}_t ∈ ⋂_{ξ ∈ 𝒫(U)} \ker ξ$, so $\dot{π}_t = h_t · (v ∘ π)$ for some $h_t ∈ 𝒞^∞(X)$.
  Take $ξ ∈ 𝒫(U)$, and a local primitive $df = ξ$, and $g ∈ 𝒞^∞(X)$. \cref{thm:integrable_system_pertubation} gives
  \begin{align*}
    0 &= \{\underbrace{df \dot{π}_t}_{=0}, g π_t\} + \{π_t^* f,dg \dot{π}_t\} = \{π_t^* f, h_t π_t^*(dg(v))\} \\
      &= π_t^*(dg v)\{fπ_t, h_t\} + h_t \underbrace{\{π_t^* f,π_t^* (dg(v))\}}_{=0}
  \end{align*}
  which implies $\{fπ_t,h_t\} = X_ξ(h_t) = 0$, meaning $h_t$ is $ξ$-invariant.
\end{proof}

\begin{proposition}
  Let $U_n$
\end{proposition}

\subsection{Nodal Trades}

Take $X = ℂ^2 × (T^*S¹)^{n-2}$, let $γ \colon ℝ^{n-2} → ℝ_{≥0}$ be a smooth function and equip $X$ with the integrable system $𝐇 \colon X → ℝ_{≥0} ×ℝ^{n-1}$ given by
\[𝐇(z_1,z_2, p_3,θ_3, … , p_n,θ_n) = (\abs{z_1 z_2 - γ(p_3, …, p_n)}^2, \abs{z_2}^2 - \abs{z_2}^2, p_3, …, p_n) \;,\]
where we take coordinates $(p,θ) ∈ ℝ × S¹ ≅ T^*S¹$.

Recall that the "standard" toric system on $ℂ²$ is given by $μ_{ℂ²}(z_1,z_2) = (\abs{z_1}^2, \abs{z_2}^2)$ and on $(T^*S¹)^{n-2}$ by $μ_{T^*S¹}(p,θ) = p$.

If $γ ≡ c$ is constant this corresponds to the product of the "Auroux system" (see e.g.\ \cite[Section 7.1]{evans2021atfs}) with $μ_{T^*S¹}^{n-2}$.
If $γ ≡ 0$ the system is toric, with corank $2$ elliptic critical values at $\{(0,0)\} × ℝ^{n-2}$, equivalent to the standard toric system $μ_{ℂ²} × μ_{T^*S¹}^{n-2}$ on $X$.

The critical values of $𝐇$ are as follows:
\begin{enumerate}
  \item $N = \{(\abs{γ}^2, 0, p), p ∈ ℝ^{n-2} \mid γ(p) > 0\}$ are corank $2$ focus-focus critical values.
  \item $\{(0,0,p), p ∈ ℝ^{n-2} \mid γ(p) = 0\}$ are corank $2$ elliptic critical values.
  \item $\{ (0,a,p), a ∈ ℝ ∖ \{0\}, p ∈ ℝ^{n-2} \}$ are corank $1$ elliptic critical values.
\end{enumerate}

We are particularly interested in the case where $γ$ is a bump function with $γ(0) > 0$ and compact support $C ⊂ ℝ^{n-2}$.
That way, on $ℝ_{≥0} × ℝ × (ℝ^{n-2} ∖ C)$, the system $𝐇$ is equivalent to the standard toric system $μ_{ℂ²} × μ_{T^*S¹}^{n-2}$.

As in \cite[Section 8.2]{evans2021atfs}, we can.

\section{Thin Branch Cut Tangles}

Let $B$ be an integral affine manifold with integral lattice $Λ ⊂ TB$.

\begin{definition}
  A \textbf{line segment} is a simple curve $γ \colon [a,b] → B$ with $ℓ'(t) ∈ Λ_{ℓ(t)}$.
\end{definition}

\begin{definition}
  \label{def:bcs}
  A \textbf{thin branch cut segment} $𝓋 = (k,v,λ)$ is a line segment $ℓ$ where $v,λ$ are primitive (co-)vector fields along $ℓ$ such that $v = ℓ'$ and $⟨λ,v⟩ = 0$.

  We call $𝓋^{-1} = (ℓ^{-1},-v,λ)$ the \textbf{reverse} of $𝓋$, where $ℓ^{-1}$ denotes the reverse of the path $ℓ$.
\end{definition}

\todo{resolution map defined on $T_xB$}
\begin{definition}
  \label{def:bcs_resolution}
  To a thin branch cut segment $𝓋 = (ℓ,v,λ)$ we associate its \textbf{local resolution map} $\overline{𝓋}_x \colon T_x B → T_x B$ for every $x ∈ \imℓ$ given by the piecewise integral linear map
  \[\overline{𝓋}_x u = 
    \begin{cases}
      u - v ⟨λ,u⟩ & \text{if } ⟨λ,u⟩ ≥ 0 \\
      u             & \text{otherwise}
    \end{cases}
  \]
  or equivalently coordinate\-/wise by the tropical rational functions
  \[(\overline{𝓋}_x u)_i = \ab(0⊕u^{-λ})^{-v_i} ⊙ u_i \; ,\]
  for any choice of identification $(T_x B, Λ_x) → (ℝ^n, ℤ^n)$.

  If $𝓊 = (\hat{ℓ},u,μ)$ is another thin branch cut segment, we write
  \[\overline{𝓋} 𝓊 = 
    \begin{cases}
      \ab([\overline{𝓋}c, \overline{𝓋}d],
      u - v ⟨λ,u⟩,
      μ + λ ⟨μ,v⟩)
      & \text{if } [c,d] ⊂ \{ x \mid ⟨λ,x-a ⟩ ≥ 0\} \\
      𝓊
      & \text{if } [c,d] ⊂ \{ x \mid ⟨λ,x-a ⟩ ≤ 0\} \\
      \text{undefined} & \text{otherwise}
    \end{cases} \;.\]
\end{definition}

\begin{remark}
  The map $μ ↦ μ + λ ⟨μ,v ⟩$ is the dual-inverse of the linear map $x ↦ x - v ⟨λ,x⟩$.
Note that $\overline{𝓋^{-1}}=\overline{𝓋}^{-1}$.

If $𝓊 = ([c,d],u,μ), 𝓋 =([a,b],v,λ)$ satisfy $[a,b] = [c,d]$ as sets, we say they \textbf{coincide}. In this case $\overline{𝓊}$ and $\overline{𝓋}$ commute.\todo{When do $𝓊$ and $𝓋$ commute?}
\end{remark}

\begin{definition}
  \label{def:stack}
  A \textbf{thin branch cut stack} $𝒮$ at $x ∈ ℝ^n$ of height $m$ is a $m$-tuple of pairs of thin branch cut segments $((𝓋_{01},𝓋_{02}),…,(𝓋_{m1},𝓋_{m2}))$, satisfying the following:
  Write $𝓋_{ij} = ([a_{ij},b_{ij}],v_{ij},λ_{ij})$.
  \begin{enumerate}
    \item $b_{i1} = a_{i2} = x$.
    \item (Top is straight) $v_{01} = v_{02}$ and $λ_{01} = λ_{02}$.
    \item (NCAS\footnote{Node collision avoidance system}) If $⟨λ_{ij},v_{kl} ⟩ = 0$ then $v_{ij}= ± v_{kl}$.
    \item (Reduction) If $m>0$ then $((\overline{𝓋_{01}} 𝓋_{11}, \overline{𝓋_{01}} 𝓋_{12}), …, (\overline{𝓋_{01}} 𝓋_{m1}, \overline{𝓋_{01}} 𝓋_{m2}))$ is a thin branch cut stack of height $m-1$.\label{itm:stack_reduce}
  \end{enumerate}

  Reducing the height of a stack as in \cref{itm:stack_reduce}, we apply the resolution map $u_0 = \overline{𝓋_{01}}$.
  Repeatedly reducing the height until the stack is empty, we successively apply resolution maps
  \[
    u_0 = \overline{𝓋_{01}}\;, \quad
    u_1 = \overline{\overline{𝓋_{01}} 𝓋_{11}}\;, \quad
    …\;, \quad
    u_m = \overline{\overline{\overline{\overline{ 𝓋_{01}} 𝓋_{11}} 𝓋_{21}} … 𝓋_{m1}} \;.
  \]
  The composition $u_m ∘ … ∘ u_1 ∘ u_0$ is called the \textbf{resolution map} of $𝒮$.
\end{definition}

\begin{definition}
  \label{def:bct}
  A \textbf{thin branch cut tangle} $𝒯 ⊂ Δ$ is an embedded directed graph $(C,B)$ where the edges $B$ are thin branch cuts segments, and every vertex $x ∈ C$ belongs to one of the following classes:
  \begin{description}
    \item[Start] $\deg(x) = (0,1)$, $x ∈ Δ^{n-2} ∖ Δ^{n-3}$ with the $n-1$-faces adjacent to $x$ having primitive normal vectors $λ_1,λ_2$ and the outgoing segment $𝓋 = ([x,b],v,λ)$ satisfies $λ = λ_1-λ_2$ and $⟨λ_1,v ⟩ = ⟨ λ_2,v ⟩ = 1$.
    \item[End] $\deg(x) = (m,0)$ and all adjacent edges coincide.
    \item[U-Turn] $\deg(x) = (m,m)$, all adjacent edges coincide and arrange into pairs $𝓋_1, 𝓋_2$ satisfying $𝓋_1^{-1} = 𝓋_2$.
    \item[Stack] $\deg(x) = (m,m)$, and the adjacent edges arrange into a thin branch cut stack of height $m$.
  \end{description}
\end{definition}


\section{Resolution Groups}

Ask Virginie Charette, William Goldman

\emergencystretch=1em
\printbibliography
\end{document}
